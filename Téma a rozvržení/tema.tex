\documentclass{article}
\usepackage{graphicx} % Required for inserting images
\usepackage[utf8]{inputenc}
\usepackage[czech]{babel} % pro správné nastavení češtiny
\usepackage[unicode={true}]{hyperref} % pro správné zobrazení obsahu

\title{Vývoj aplikace pro správu členské databáze v C\#}
\author{Filip Mach}
\date{Říjen 2025}

\begin{document}
\maketitle

\section{Úvod}
Cílem práce by mělo být vytvoření desktopové aplikace, jež by umožňovala „ukládat“ a získávat 
komplexní informace o členech určitého spolku. Základem by měla být databáze členů se základními
údaji o nich, dostupná administrátorovi respektive vybraným uživatelům. 

Vize je taková, že existuje „vedoucí“ určitého spolku (např. sportovního klubu), který potřebuje
vést členskou základnu se základními údaji (jméno, datum narození atd.) o každém členovi. 
Dále potřebuje zaznamenávat docházku určitých členů, nebo třeba příslušnost do určitých družstev,
či tříd. Tato aplikace by mu měla umožnit všechny tyto informace bezpečně uložit, mít k nim
prostřednictvím aplikace přístup a možnost je upravovat. Celkově si to lze představit jako takové
Bakaláře.

Motivace za tímto tématem je taková, že mě požádal konkrétní sportovní klub o vytvoření 
takovéto aplikace.

\section{Teoretická část}
Požadované znalosti pro vytvoření této aplikace by měly být znalost nějakého programovacího 
jazyka pro vytvoření desktopové aplikace, v mém případě C\#. Dále znalost jazyku pro vytvoření 
databáze a komunikaci s ní (SQL) a schopnost komunikace mezi těmito dvěma části (frontend a 
backend).

Základní principem by tedy mělo být určité spojení databáze s desktopovou aplikací. Pro komunikaci
s databází využiji knihovnu Entity Framework Core. Co se týče zabezpečení dat, tak je hlavní
aby všechna citlivá data byly v databázi zašifrované a veškerá komunikace mezi frontendem a 
backendem nepřenášela žádná nezašifrovaná citlivá data. Tomuto napomůže i využití takzvaného 
MVVM patternu, který efektivně oddělí UI od samotné databáze.\\
Aplikace fungující na podobném principu již existují, ovšem všechny dostačující pro naše potřeby 
jsou nepřiměřeně zpoplatněné, tudíž se nabízí si vytvořit aplikace vlastní.

\newpage

\section{Praktická část}
Aplikace by  měla mít následující funkcionality:
\begin{itemize}
    \item možnost zobrazit a upravovat informace o každém členovy (datum narození, družstvo atd.)
    \item jednoduché zapisování docházky na konkrétní tréninky (či schůze)
    \item omezení přístupu k jednotlivým informacím pro určité uživatele
    \item možnost komunikace administrátora s členy
    \item možnost přihlášení se jednotlivých členů do aplikace, pro zobrazení informací o nich 
    samotných
\end{itemize}
Aplikace by měla být navržená tak, aby šlo v budoucnu přidat libovolné další funkcionality, 
případně spojit ji s webovou stránkou či mobilní aplikací. Nejsem si jistý, co všechno lze 
stihnout v rámci této ročníkové práce, je teda možné, že jednotlivé funkcionality přidám, či
vynechám.

\section{Závěr}
Měla by vzniknout v reálu použitelná aplice, jež umožní jednoduchou práci v rámci výše 
definovaných funkcionalit.

\section{Výstup}
Práce v minimálním rozsahu 10 stran textu, prezentace pro představení práce a samotná aplikace.
\end{document}

